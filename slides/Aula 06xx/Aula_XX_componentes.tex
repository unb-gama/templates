
\subsection{Inversores}

\begin{frame}{Inversores}

\textbf{Operação}
\begin{itemize}
\item Funciona unicamente quando está conecto à rede elétrica

\item Possui um esquema anti-ilhamento (segurança)
\end{itemize}
 
\begin{figure}[H]
\includegraphics[scale=.28,center]{16}
\end{figure}

\end{frame}

\begin{frame}{Inversores}

\begin{columns}[T]
    \begin{column}{0.5\textwidth}
    \vspace{.5cm}
    	\textbf{Características}
    \end{column}
    \begin{column}{0.5\textwidth}
    	\centering
      	\includegraphics[scale=.2,right]{17}
    \end{column}
\end{columns}
 
\begin{figure}[H]
\includegraphics[scale=.33,center]{18}
\end{figure}

\end{frame}

%%%%%%%%%%%%%%%%%%%%%%%%%%%%%%%%%%%	32	%%%%%%%%%%%%%%
%%%%%%%%%%%%%%%%%%%%%%%%%%%%%%%%%%%%%%%%%%%%%%%%%%%%%
\begin{frame}{Inversores}

\begin{columns}[T]
    \begin{column}{0.5\textwidth}
    \vspace{.5cm}
    	\textbf{Características}
    \end{column}
    \begin{column}{0.5\textwidth}
    	\centering
      	\includegraphics[scale=.2,right]{19}
    \end{column}
\end{columns}
 
\begin{figure}[H]
\includegraphics[scale=.33,center]{20}
\end{figure}

\end{frame}


\subsection{Medidor Bidirecional}

\begin{frame}{Medidor Bidirecional de Energia}

Indica o sentido de fluxo da energia

\vspace{.35cm}
\centering
{\Large fonte <=> carga}

\vspace{.35cm}

\begin{columns}[T]
    \begin{column}{0.5\textwidth}
    	\centering
      	\includegraphics[scale=.35,center]{21}
    \end{column}
    \begin{column}{0.5\textwidth}
    	\centering
      	\includegraphics[scale=.45,center]{22}
    \end{column}
\end{columns}
 
\end{frame}

%%%%%%%%%%%%%%%%%%%%%%%%%%%%%%%%%%%	34	%%%%%%%%%%%%%%
%%%%%%%%%%%%%%%%%%%%%%%%%%%%%%%%%%%%%%%%%%%%%%%%%%%%%
\begin{frame}{Medidor Bidirecional de Energia}

O gráfico a seguir indica os quatro quadrantes de acordo com os valores de energia

\begin{figure}[H]
\includegraphics[scale=.45,center]{23}
\end{figure}

\vspace{-.5cm}
\begin{figure}[H]
\includegraphics[scale=.45,center]{24}
\end{figure}

\end{frame}

%%%%%%%%%%%%%%%%%%%%%%%%%%%%%%%%%%%	35	%%%%%%%%%%%%%%
%%%%%%%%%%%%%%%%%%%%%%%%%%%%%%%%%%%%%%%%%%%%%%%%%%%%%
\section{Exemplo}
\subsection*{Exemplo}


\begin{frame}{Exemplo}

Dimensionar um sistema fotovoltaico para atender o consumo diário de um pequeno escritório que possui os seguintes aparelhos

\begin{itemize}
\item Lâmpada fluorescente compacta - 11 w
\item Modem de internet
\item Roteador
\item Impressora 
\end{itemize}

\end{frame}
%%%%%%%%%%%%%%%%%%%%%%%%%%%%%%%%%%%	36	%%%%%%%%%%%%%%
%%%%%%%%%%%%%%%%%%%%%%%%%%%%%%%%%%%%%%%%%%%%%%%%%%%%%%
\begin{frame}{Exemplo}

\textbf{Condições do sistema
}

\begin{itemize}
\item Baterias de chumbo ácido de 12 Vcc
\item Tensão do banco de baterias de 24 Vcc
\item Descarga máxima de 50%
\item Controlador de carga convencional
\item Capacidade de armazenamento para dois dias
\item Módulos Solares Bosch c-Si M 60 – Designação 250
\item Tensão da instalação elétrica 127 Vac
\end{itemize}

\end{frame}

%%%%%%%%%%%%%%%%%%%%%%%%%%%%%%%%%%%	37	%%%%%%%%%%%%%%
%%%%%%%%%%%%%%%%%%%%%%%%%%%%%%%%%%%%%%%%%%%%%%%%%%%%%%
\begin{frame}{Exemplo}

\textbf{Organização do sistema
}

\begin{itemize}
\item Quatro baterias de 240 Ah / 24V
\item Cinco módulos solares Bosch c-Si M 60 – Designação 250
\item Um controlador de carga 24 V / 60 A
\item Um inversor 24 Vcc / 127 Vac

\end{itemize}

\end{frame}



\begin{frame}{Dúvidas?}

\end{frame}



